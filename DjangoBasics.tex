\documentclass[12pt]{beamer}

% list of packages used
\usepackage[latin1]{inputenc}
\usepackage{amsmath}
\usepackage{amsfonts}
\usepackage{amssymb}
\usepackage{graphicx}
\usepackage{listings}

% Author details
\author{Sanyam, Ashutosh @EduTechLabs}
\title{Django Basics}

% configs

%%set theme 
\usetheme{Luebeck}

%%configure listings package for python
% Default fixed font does not support bold face
\DeclareFixedFont{\ttb}{T1}{txtt}{bx}{n}{8pt} % for bold
\DeclareFixedFont{\ttm}{T1}{txtt}{m}{n}{8pt}  % for normal

%custom colours
\usepackage{color}
\definecolor{deepblue}{rgb}{0,0,0.5}
\definecolor{deepred}{rgb}{0.6,0,0}
\definecolor{deepgreen}{rgb}{0,0.5,0}

%configuring listings package to highlight python syntax
\newcommand\pythonstyle{\lstset{
language=Python,
basicstyle=\ttm,
otherkeywords={self,plot,run},             % Add keywords here
keywordstyle=\ttb\color{deepblue},
emph={MyClass,__init__},          % Custom highlighting
emphstyle=\ttb\color{deepred},    % Custom highlighting style
stringstyle=\color{deepgreen},
%frame=tb,                         % Any extra options here
showstringspaces=false            % 
}}

% Python environment
\lstnewenvironment{python}[1][]
{
\pythonstyle
\lstset{#1}
}
{}

% Python for external files
\newcommand\pythonexternal[2][]{{
\pythonstyle
\lstinputlisting[#1]{#2}}}

% Python for inline
\newcommand\pythoninline[1]{{\pythonstyle\lstinline!#1!}}

%%%%%%%%%%%% begin writing the actual document %%%%%%%%%%%%%%%%%
\begin{document}

%get the title page
\frame{\maketitle}
% welcome page 
\begin{frame}{Welcome to the course}
	\begin{center}
	 Welcome to \emph{\color{deepblue}EduTechLab's} Basic Django Course.
	\end{center}
\end{frame}

%%%%%%%%%%%%%%% about python

% preparing the code

% preparing the slide
\begin{frame}[containsverbatim]{Django At Glance}
	\begin{flushleft}
	\begin{itemize}
		\item{Designed to make Web Development} \\
		
		\item{Common useDatabase Driven Web-apps}\\

		\item{don't need to write sql quries} \\
	
	\end{itemize}
	\end{flushleft}
\end{frame}

\begin{frame}[containsverbatim]{Install virtual environment}
	\begin{flushleft}
	\begin{itemize}
		\item{Install python setup tool} \\
		\pythoninline{sudo apt-get install python-setuptool}
		\item{Install Virtual Environment}\\
		\pythoninline{sudo easy_install virtualenv}
		\item{Setting up Defalt behaviour like..} \\
		\pythoninline{virtualenv --no-site-packages django-projectname }
		\item{Activate the Environment} \\
		\pythoninline{source django-projectname/bin/activate}
		\item{Install Django}\\
		\pythoninline{easy_install django}
	
	\end{itemize}
	\end{flushleft}
\end{frame}

\begin{frame}[containsverbatim]{Creation Of Project }
	\begin{flushleft}
	\begin{itemize}
		\item{create project first by running} \\
		\pythoninline{django-admin.py startproject ProjectName}
		\item{create application to be run inside that ProjectName directory}\\
		\pythoninline{django-admin.py startapp ApplicationName}
		\item{activate Virtual Environment} \\
		\pythoninline{source projectname/bin/activate}
			
	\end{itemize}
	\end{flushleft}
\end{frame}

\begin{frame}[containsverbatim]{Details Of Project Folder}
	\begin{itemize}
        \item{ProjectName Folder contains : } 
		\begin{itemize}
			\item{ProjectName folder}
			\item{manage.py}
			\item{ApplicationName Folder}
			\item{Template Folder}
			\item{DatabaseName.db}
		\end{itemize}
	\end{itemize}
\end{frame}

\begin{frame}[containsverbatim]{Details Of ApplicationName Folder}
	\begin{itemize}
        \item{ApplicationName Folder contains :} 
		\begin{itemize}
			\item{models.py}
			\item{views.py}
			\item{urls.py}
%			\item{\__init\__.py}
			\item{wsgi.py}
		\end{itemize}
	\end{itemize}
\end{frame}

\begin{frame}[containsverbatim]{Details Of Models.py}
		\pythoninline{models.py Contains :- }
		\begin{python}
		from django.db import models
		  class Article(models.Model):
			pub_date = models.DateField()
			headline = models.CharField(max_length=200)
			content = models.TextField()
 			reporter = models.ForeignKey(Reporter)
 		\end{python}			
\end{frame}

\begin{frame}[containsverbatim]{Entry of Database}
	\begin{itemize}
		\pythoninline{Make Entry in settings.py :-}
		\begin{python}
DATABASES = {
	 default': {
      'ENGINE': 'django.db.backends.sqlite3',
      'NAME': os.path.join(BASE_DIR, 'db.sqlite3'),
    			}
		}
		\end{python}					
		\item{Absolute Path :- }
		\pythoninline{os.path.join}	
		\item{Relative path :-}
		\pythoninline{/home/Username/ApplicationFolder/ApplicationFolder/DatabaseName.db}
	\end{itemize}		
\end{frame}

\begin{frame}[containsverbatim]{AppName Setting}
	\pythoninline{making entry of AppName in settins.py :-}
	\begin{python}
	
		INSTALLED_APPS = (
		    'django.contrib.admin',
		    'django.contrib.auth',
		    'django.contrib.contenttypes',
		    'django.contrib.sessions',
		    'django.contrib.messages',
		    'django.contrib.staticfiles',
		    'ApplicationName',
		)
	\end{python}
\end{frame}

\begin{frame}[containsverbatim]{Synchronization}
	\begin{itemize}
		\item{Synchronize database}
		\newline
		\pythoninline{python manage.py syncdb }	
	\end{itemize}
\end{frame}

\begin{frame}[containsverbatim]{Detail Of Urls}
	\begin{itemize}
		\item{Setting Up Url}
		\begin{python}
from django.conf.urls import patterns, include, url
 urlpatterns = patterns('',
 url  (r'^urlpattern/', 
          include('ApplicationName.ApplicationName_urls')),
	)
		\end{python}
	\end{itemize}
\end{frame}

\begin{frame}[containsverbatim]{Template}
	\begin{itemize}
		\item{Simple Template}
		  \begin{python}
      	 <html>
        <body>
           Hi {{ name }} ,
       this is the first demonstraion for the danjo project 
   		   </body>
    		  </html> 
		  \end{python}
	\end{itemize}
\end{frame}

\begin{frame}[containsverbatim]{Views}
	\begin{itemize}
		\item{Simple Views.py}
		  \begin{python}
	#from django.shortcuts import render
	#from django.views.generic.base import TemplateView
     
     def hello_template_simple(request):
		name = "eduTech"
	return render_to_response('hello.html',{'name':name})
		  \end{python}
	\end{itemize}
\end{frame}

\begin{frame}[containsverbatim]{Forms}
	\begin{itemize}
		\item{Simple Forms}
		  \begin{python}
	from django import forms	
	from models import Newclass
	   class FormName(forms.ModelForm):
		class Meta:
			model = Newclass
			fields = ('title', 'body', 'pub_date')
		  \end{python}
	\end{itemize}
\end{frame}

\begin{frame}[containsverbatim]{Sell Entery}
	\begin{itemize}
		\item{Insert data throgh Shell}
		  \begin{python}
	from ApplicationName.models import ClassName
		object = ClassName(attribute='value', . . .) 
		object.save()
		  \end{python}
	\end{itemize}
\end{frame}	

\begin{frame}[containsverbatim]{RunServer}
	\begin{itemize}
		\item{Runserver}
			\newline
			\pythoninline{python manage.py runserver}
		\item{Combination Of Ip and Port}
			\newline
			\pythoninline{127.0.0.1:8000}
	\end{itemize}
\end{frame}

\end{document}
